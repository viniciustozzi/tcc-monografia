\setlength{\absparsep}{18pt} % ajusta o espaçamento dos parágrafos do resumo
\begin{resumo}[Abstract]
The current project presents an IoT system that uses concept of V2V and V2I to help emergency units at the traffic, letting drivers know they are commuting through the same path of an emergency unity and that they need to make the path clear for that unit to get to its destination faster. And beyond that the project stores all the data gathered from that process for further analysis of the Big Data, so that we can measure some indicators that may help to improve the health and safety of the traffic. To accomplish it was used new tools and a new architecture that is scalable, robust, easy to use and it is gonna be capable of integrate with another systems that may be created in the future. The stress test report and simulations show that those objectives were achieved, it behaved satisfactorily, and achieved high data throughput . In summary it can be said that this architecture reached its goal using agile system like Docker and Clojure.

 \textbf{Keywords}: IoT. Big Data. Clojure.
\end{resumo}